\documentclass[12pt]{article}
\usepackage[utf8]{inputenc}
\usepackage[a6paper]{geometry}
\usepackage[LGR,T2A]{fontenc}
\usepackage[russian]{babel}
\usepackage{indentfirst}
\usepackage{tocloft}
\usepackage{fancyhdr}
\usepackage{calc}
\usepackage{xcolor}
\usepackage{tikz}
\usepackage{titling}
\usepackage{hyperref}
\usepackage{titlesec}


\newcommand{\rom}[1]{\uppercase\expandafter{\romannumeral #1\relax}}

\title{Право и государство}
\author{Людвиг Андреас фон Фейербах}
\date{}

\tolerance=10000
\hbadness=10000
\vbadness=10000
\clubpenalty=10000
\widowpenalty=10000
\displaywidowpenalty=10000
\interfootnotelinepenalty=10000

\begin{document}

\maketitle

\newpage

Я живу не потому, что я имею право жить, но потому я и имею неоспоримое право жить, что живу. Право есть нечто вторичное; праву предшествует то, что не является правом, т.е. то, что есть больше права, что не является человеческим установлением.

\bigskip

Истина и достоверность права опираются только на истинность и достоверность чувств. Ими одними только оценивается подлинность подписи, подлинность клейма, подлинность монеты --- от них зависит публичный кредит. Тождество личности, сознания! Но может ли оно быть без тождества тела, которое одно только является чувственно удостоверенной вещью?

\bigskip

Право первоначально не зависит от закона, а, наоборот, закон зависит от права. Закон закрепляет только то, что является правом и по праву, только превращает право в долг для других.

\bigskip

Величайшее нравственное зло возникает потому, что люди затушевывают различие между собой и другими, различие в тождестве. Разумеется, мы оба люди; но это самое малое, это разумеется само собой. Равенство следует выдвигать против высокомерия, которое не позволяет другому быть человеком, которое считает себя выше него стоящим, преимущественным существом, существом особого рода, особого племени, как некогда дворянство считало себя в отношении к плебсу, к бюргерству; стало быть, равенство следует выдвигать против искусственного различия.

\bigskip

Ни политика, ни государство для себя самих не являются целью. Государство растворяется в людях, существует только по воле людей. Субъективный человек, так называемый субъективный человек --- вот истинный человек, истинный дух. Это истина христианства. Мы не возвращаемся назад из христианства к языческой государственной жизни, где человек растворялся в государстве, где гражданин стоял выше человека в его целостности, хотя, конечно, даже гражданин был более человеческим и идеальным существом, чем современный <<подданный>>.

\bigskip

Отличие от христианства может состоять только в том, что субъективный человек наполняется содержанием реального мира, в том, что небесная, супранатуралистическая субъективность становится практической субъективностью.

\bigskip

Свобода состоит не в чем ином, как в том, чтобы доставить человеку неограниченную сферу действий, соответствующую его целостности, всем его силам и способностям. Если государство в отличие от субъективного духа выставляется как объективно истинное, то человек деградирует до степени машины, обесчеловечивается, приносится в жертву государству как абстрактное количество. То, чем человек является в мнении, то, чем он представляется, полагается выше того, чем он является в действительности.

\bigskip

В государстве, где все зависит от милости и произвола самодержца, каждое правило становится шатким, из души с корнем вырывается представление о <<вечном нравственном законе>>, убеждение в необходимости добродетели; вырывается убеждение в необходимости строгой справедливости, не делающей ни для кого исключения; вырывается чувство самостоятельности, мужество и стремление к добродетели. Неограниченная монархия --- это безнравственное государство.

\bigskip

История --- это исключительно процесс очеловечивания человечества; первое и ближайшее к человеку как таковому является последним и отдаленнейшим. То, что человек выражает свою сущность посредством опредмечивания, рассматривая ее сначала как отличную от себя и над собой пребывающую сущность, прежде чем он начинает рассматривать ее, как свою сущность, --- что этот путь правильный --- тому история представляет тривиальнейшие примеры. То, что для католицизма было божественным учреждением, то для протестантизма стало человеческим установлением.

\bigskip

Положение: <<мир --- от Бога>>, одинаково с положением: <<король --- от Бога>>. Сколь истинно королевство божией милостью, столь же истин и мир божией милостью. Там на место естественного опосредствования, условия, причины, здесь на место политического опосредствования, условия, причины ставится воображаемая причина.

\bigskip

Путь истории человечества, конечно, есть путь, предназначенный ему, потому что человек следует движению природы, как он следует, например, движению потока. Люди тянутся туда, где они находятся место и притом место, им соответствующее. Они локализуются, они определяются местом, в котором они живут. Сущность Индии --- это сущность индийца. Индиец есть то, что он ecть, и чем он стал только в качестве продукта индийского солнца, индийского воздуха, индийских вод, индийских животных и растений. Каким путем, следовательно, мог бы человек произойти первоначально не из природы? Люди, которые первоначально приспособляются ко всякой природе, возникли из природы, которая не терпела крайности.

\bigskip

Дуализм, раздвоение --- это сущность теологии, раздвоение же является и сущностью монархии. Там мы имеем противоположность Бога и мира, здесь --- противоположность государства и народа. Там, как и здесь, собственная сущность противостоит человеку как другая сущность; там --- в качестве всеобщей сущности, здесь --- в качестве действительного, личного, или индивидуального, существа. <<Государи суть боги>>, т.е. существа, которые кажутся чем-то иным, что они суть в действительности, --- существа, которые на деле не отличаются от других людей, в воображении же почитаются существами другого, высшего, рода.

\bigskip

Воображение --- это сила теологии, и воображение же --- сила монархии. До тех пор государи будут господствовать над человечеством, пока воображение господствует над ним. Роскошь, помпа, блеск, видимость --- на одной стороне; нищета, нужда, бедность --- на другой --- вот необходимые атрибуты монархии. Сила воображения находит удовольствие и раскрывается только в превосходных степенях; высшему счастью соответствует только глубочайшее несчастье, небу --- только ад, Богу --- только дьявол.

\bigskip

Превращение теологии в антропологию в области мышления --- это превращение монархии в республику в области практики и жизни.

\bigskip

Свобода, конечно, есть самое высшее, но столь же мало, как и идея, она есть начало; она --- цель; не физическая, прирожденная способность --- человек не рожден свободным; она есть результат образования, конечно, на основе соответствующих врожденных дарований.

\bigskip

Нет ничего более смешного, как верить в то, что люди стали бы несвободными вследствие веры в необходимость человеческих волевых поступков или стали бы свободными благодаря метафизическому учению о свободе.

\bigskip

Каким образом согласуется с естественной необходимостью свободы беззаконность фантазии, заблуждения, отклонения от необходимости? Этот упрек так же нелеп, как если бы той истине, что движения животных следуют законам рычага, механики, был противопоставлен вопрос: как же согласуются с этой законосообразностью прыжки и скачки животных, беганье туда и сюда, падение и спотыкание.

\bigskip

Свобода, как и все подобные общие слова, берется в таком неопределенном смысле, что для многих отрицание свободы, т.е. фантастической свободы, тождественно с отрицанием даже произвольной перемены места, так что для них выражение: <<человек не свободен>> --- это все равно, что <<человек не есть человек, т.е. двигающееся существо, а лишь растение, камень>>.

\bigskip

Я не понимаю, каким образом может идеалист или спиритуалист, если он по крайней мере последователен, ставить своей целью внешнюю политическую свободу. Ведь спиритуалисту достаточно духовной свободы; чем больше давление извне, тем больше он имеет оснований пользоваться, напротив, внутренней свободой. Политическая свобода в понимании спиритуалистов --- это материализм в области политики. К действительной свободе на самом деле относится также и свобода материальная, телесная. Свобода печати дает простор и воздух не только моей голове, но и моему сердцу, моим легким, моей желчи. Спиритуалисту же достаточно мысленной свободы.

\bigskip

Из речи Кастеляра против испанской монархии: <<История человечества есть постоянная борьба между идеями и интересами; последние всегда побеждают на мгновение, длительно же побеждают всегда идеи>>. Что за противопоставление? Разве идеи не являются также интересами? Интересами, на мгновение только не узнанными, презираемыми, еще не действительными, не признанными законом, интересами, противоречащими особым интересам отдельных, ныне господствующих, сословий, интересами, пока еще существующими только в идее, всеобщими человеческими интересами? Не является ли справедливость общим интересом --- интересом тех, с кем поступают несправедливо, хотя бы, что разумеется само собой, и не интересом тех, кто пользуется этой несправедливостью, т.е. не интересом сословий и классов, находящих удовлетворение только в преимуществах перед другими классами. Короче говоря, борьба между идеями и интересами представляет собой борьбу между старым и новым.

\bigskip

Сколько социальных зол могло бы, тем не менее, быть устранено так легко или, по меньшей мере, не приводя к большим расстройствам, как, например, недостатки на эмигрантских кораблях, хотя они устраняются только тогда, когда достигли такого пункта, что продолжение старой практики стало совершенно невозможно. Так всегда только под давлением, всегда только по принуждению поступает человек, так поступают и правительства и никогда не поступают самопроизвольно, как можно было бы ожидать по их красивым речам; никогда не поступают свободно, разумно и предусмотрительно.

\bigskip

Какая разница между народом и чернью? Если чернь верит или делает то, что нравится или представляется полезным господствующим, то чернь является народом; в противоположном случае народ является чернью.

\end{document}
